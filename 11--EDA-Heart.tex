% Options for packages loaded elsewhere
\PassOptionsToPackage{unicode}{hyperref}
\PassOptionsToPackage{hyphens}{url}
%
\documentclass[
]{article}
\title{2.1-Análisis exploratorio de datos (EDA)}
\author{Jair Villanueva}
\date{4/4/2022}

\usepackage{amsmath,amssymb}
\usepackage{lmodern}
\usepackage{iftex}
\ifPDFTeX
  \usepackage[T1]{fontenc}
  \usepackage[utf8]{inputenc}
  \usepackage{textcomp} % provide euro and other symbols
\else % if luatex or xetex
  \usepackage{unicode-math}
  \defaultfontfeatures{Scale=MatchLowercase}
  \defaultfontfeatures[\rmfamily]{Ligatures=TeX,Scale=1}
\fi
% Use upquote if available, for straight quotes in verbatim environments
\IfFileExists{upquote.sty}{\usepackage{upquote}}{}
\IfFileExists{microtype.sty}{% use microtype if available
  \usepackage[]{microtype}
  \UseMicrotypeSet[protrusion]{basicmath} % disable protrusion for tt fonts
}{}
\makeatletter
\@ifundefined{KOMAClassName}{% if non-KOMA class
  \IfFileExists{parskip.sty}{%
    \usepackage{parskip}
  }{% else
    \setlength{\parindent}{0pt}
    \setlength{\parskip}{6pt plus 2pt minus 1pt}}
}{% if KOMA class
  \KOMAoptions{parskip=half}}
\makeatother
\usepackage{xcolor}
\IfFileExists{xurl.sty}{\usepackage{xurl}}{} % add URL line breaks if available
\IfFileExists{bookmark.sty}{\usepackage{bookmark}}{\usepackage{hyperref}}
\hypersetup{
  pdftitle={2.1-Análisis exploratorio de datos (EDA)},
  pdfauthor={Jair Villanueva},
  hidelinks,
  pdfcreator={LaTeX via pandoc}}
\urlstyle{same} % disable monospaced font for URLs
\usepackage[margin=1in]{geometry}
\usepackage{color}
\usepackage{fancyvrb}
\newcommand{\VerbBar}{|}
\newcommand{\VERB}{\Verb[commandchars=\\\{\}]}
\DefineVerbatimEnvironment{Highlighting}{Verbatim}{commandchars=\\\{\}}
% Add ',fontsize=\small' for more characters per line
\usepackage{framed}
\definecolor{shadecolor}{RGB}{248,248,248}
\newenvironment{Shaded}{\begin{snugshade}}{\end{snugshade}}
\newcommand{\AlertTok}[1]{\textcolor[rgb]{0.94,0.16,0.16}{#1}}
\newcommand{\AnnotationTok}[1]{\textcolor[rgb]{0.56,0.35,0.01}{\textbf{\textit{#1}}}}
\newcommand{\AttributeTok}[1]{\textcolor[rgb]{0.77,0.63,0.00}{#1}}
\newcommand{\BaseNTok}[1]{\textcolor[rgb]{0.00,0.00,0.81}{#1}}
\newcommand{\BuiltInTok}[1]{#1}
\newcommand{\CharTok}[1]{\textcolor[rgb]{0.31,0.60,0.02}{#1}}
\newcommand{\CommentTok}[1]{\textcolor[rgb]{0.56,0.35,0.01}{\textit{#1}}}
\newcommand{\CommentVarTok}[1]{\textcolor[rgb]{0.56,0.35,0.01}{\textbf{\textit{#1}}}}
\newcommand{\ConstantTok}[1]{\textcolor[rgb]{0.00,0.00,0.00}{#1}}
\newcommand{\ControlFlowTok}[1]{\textcolor[rgb]{0.13,0.29,0.53}{\textbf{#1}}}
\newcommand{\DataTypeTok}[1]{\textcolor[rgb]{0.13,0.29,0.53}{#1}}
\newcommand{\DecValTok}[1]{\textcolor[rgb]{0.00,0.00,0.81}{#1}}
\newcommand{\DocumentationTok}[1]{\textcolor[rgb]{0.56,0.35,0.01}{\textbf{\textit{#1}}}}
\newcommand{\ErrorTok}[1]{\textcolor[rgb]{0.64,0.00,0.00}{\textbf{#1}}}
\newcommand{\ExtensionTok}[1]{#1}
\newcommand{\FloatTok}[1]{\textcolor[rgb]{0.00,0.00,0.81}{#1}}
\newcommand{\FunctionTok}[1]{\textcolor[rgb]{0.00,0.00,0.00}{#1}}
\newcommand{\ImportTok}[1]{#1}
\newcommand{\InformationTok}[1]{\textcolor[rgb]{0.56,0.35,0.01}{\textbf{\textit{#1}}}}
\newcommand{\KeywordTok}[1]{\textcolor[rgb]{0.13,0.29,0.53}{\textbf{#1}}}
\newcommand{\NormalTok}[1]{#1}
\newcommand{\OperatorTok}[1]{\textcolor[rgb]{0.81,0.36,0.00}{\textbf{#1}}}
\newcommand{\OtherTok}[1]{\textcolor[rgb]{0.56,0.35,0.01}{#1}}
\newcommand{\PreprocessorTok}[1]{\textcolor[rgb]{0.56,0.35,0.01}{\textit{#1}}}
\newcommand{\RegionMarkerTok}[1]{#1}
\newcommand{\SpecialCharTok}[1]{\textcolor[rgb]{0.00,0.00,0.00}{#1}}
\newcommand{\SpecialStringTok}[1]{\textcolor[rgb]{0.31,0.60,0.02}{#1}}
\newcommand{\StringTok}[1]{\textcolor[rgb]{0.31,0.60,0.02}{#1}}
\newcommand{\VariableTok}[1]{\textcolor[rgb]{0.00,0.00,0.00}{#1}}
\newcommand{\VerbatimStringTok}[1]{\textcolor[rgb]{0.31,0.60,0.02}{#1}}
\newcommand{\WarningTok}[1]{\textcolor[rgb]{0.56,0.35,0.01}{\textbf{\textit{#1}}}}
\usepackage{graphicx}
\makeatletter
\def\maxwidth{\ifdim\Gin@nat@width>\linewidth\linewidth\else\Gin@nat@width\fi}
\def\maxheight{\ifdim\Gin@nat@height>\textheight\textheight\else\Gin@nat@height\fi}
\makeatother
% Scale images if necessary, so that they will not overflow the page
% margins by default, and it is still possible to overwrite the defaults
% using explicit options in \includegraphics[width, height, ...]{}
\setkeys{Gin}{width=\maxwidth,height=\maxheight,keepaspectratio}
% Set default figure placement to htbp
\makeatletter
\def\fps@figure{htbp}
\makeatother
\setlength{\emergencystretch}{3em} % prevent overfull lines
\providecommand{\tightlist}{%
  \setlength{\itemsep}{0pt}\setlength{\parskip}{0pt}}
\setcounter{secnumdepth}{-\maxdimen} % remove section numbering
\ifLuaTeX
  \usepackage{selnolig}  % disable illegal ligatures
\fi

\begin{document}
\maketitle

\hypertarget{anuxe1lisis-exploratorio-de-datos-eda}{%
\subsubsection{Análisis Exploratorio de datos
(EDA)}\label{anuxe1lisis-exploratorio-de-datos-eda}}

\textbf{Definición:}

Son técnicas estadísticas cuya finalidad es conseguir un entendimiento
básico de los datos y de las relaciones existentes entre las variables
analizadas de un dataset.

\textbf{Objetivos del EDA}

\begin{itemize}
\tightlist
\item
  Organizar y preparar los datos
\item
  Detectar fallos en el diseño y recogida de los datos
\item
  Manejo de datos faltantes
\item
  Transformar/eliminar/agregar/ordenar datos
\item
  Identificación de casos atípicos (outliers)
\item
  Encontrar relaciones entre variables
\item
  Aplicar técnicas estadísticas para cada variable (Gráficas)
\end{itemize}

\hypertarget{introducciuxf3n}{%
\subsubsection{Introducción}\label{introducciuxf3n}}

Las enfermedades cardiovasculares (ECV) o enfermedades del corazón son
la principal causa de muerte en todo el mundo con 17,9 millones de casos
de muerte cada año . Las enfermedades cardiovasculares se deben a la
hipertensión, la diabetes, el sobrepeso y los estilos de vida poco
saludables. Puede leer más sobre las estadísticas de enfermedades
cardíacas y las causas de la autocomprensión. Este proyecto cubre el
análisis exploratorio de datos usando R. El conjunto de datos utilizado
en este proyecto es el conjunto de datos de enfermedades cardíacas
(``heart.csv''), y tanto los datos como el código para este proyecto
están disponibles en mi repositorio de GitHub
\href{Ew3006-DataSciencexHC}{https://github.com/jairvilla/EW3006\_DataSciencexHC.git}.

\textbf{Información de las variables del dataset:}

1)\textbf{age:} Age of the patient, in years 2)\textbf{sex:} (1= male, 0
= female) 3)\textbf{cp:} chest pain type (4 values) Value 1: typical
angina Value 2: atypical angina Value 3: non-anginal pain Value 4:
asymptomatic 4)\textbf{trestbps} resting blood pressure 5)\textbf{chol:}
serum cholestoral in mg/dl 6)\textbf{fbs:} fasting blood sugar
\textgreater{} 120 mg/dl (1 = true, 0 = false) 7)\textbf{restecg:}
resting electrocardiographic results (values 0,1,2) Value 0:normal,
Value 1: having ST-T wave abnormality, Value 2: showing probable or
definite left ventricular hypertropy by Estes 8)\textbf{thalach:}
maximum heart rate achievecd 9)\textbf{exang:} exercise induced angina
(1 = yes, 0 = no) 10)\textbf{oldpeak:} = ST depression induced by
exercise relative to rest 11)\textbf{slope:} the slope of the peak
exercise ST segment 12).ca: number of major vessels (0-3) colored by
flourosopy 13\textbf{thal:} 3 = normal; 6 = fixed defect; 7 = reversable
defect 14)\textbf{target:} Diagnoses of heart disease (Value 0:
\textless50\% diameter narrowing, Value 1: \textgreater{} 50\% diameter
narrowing)

\hypertarget{cargar-el-dataset-heart.csv}{%
\subsubsection{\texorpdfstring{Cargar el dataset:
\emph{``heart.csv''}}{Cargar el dataset: ``heart.csv''}}\label{cargar-el-dataset-heart.csv}}

\begin{Shaded}
\begin{Highlighting}[]
\NormalTok{heart }\OtherTok{\textless{}{-}} \FunctionTok{read.csv}\NormalTok{(}\StringTok{"./data/heart.csv"}\NormalTok{, }\AttributeTok{sep =} \StringTok{";"}\NormalTok{, }\AttributeTok{header =} \ConstantTok{TRUE}\NormalTok{ )}
\CommentTok{\# glimpse(heart)   \# Resumir los datos}
\FunctionTok{summary}\NormalTok{(heart)     }\CommentTok{\# Estadistica descriptiva}
\end{Highlighting}
\end{Shaded}

\begin{verbatim}
##        id             age             sex               cp       
##  Min.   :  1.0   Min.   : 2.00   Min.   :0.0000   Min.   :1.000  
##  1st Qu.: 76.5   1st Qu.:47.00   1st Qu.:0.0000   1st Qu.:3.000  
##  Median :152.0   Median :55.00   Median :1.0000   Median :3.000  
##  Mean   :152.0   Mean   :54.11   Mean   :0.6799   Mean   :3.158  
##  3rd Qu.:227.5   3rd Qu.:61.00   3rd Qu.:1.0000   3rd Qu.:4.000  
##  Max.   :303.0   Max.   :77.00   Max.   :1.0000   Max.   :4.000  
##                  NA's   :10                                      
##     trestbps           chol            fbs            restecg      
##  Min.   :  94.0   Min.   :126.0   Min.   :0.0000   Min.   :0.0000  
##  1st Qu.: 120.0   1st Qu.:212.0   1st Qu.:0.0000   1st Qu.:0.0000  
##  Median : 130.0   Median :243.0   Median :0.0000   Median :1.0000  
##  Mean   : 145.9   Mean   :247.8   Mean   :0.1485   Mean   :0.9901  
##  3rd Qu.: 140.0   3rd Qu.:277.0   3rd Qu.:0.0000   3rd Qu.:2.0000  
##  Max.   :2000.0   Max.   :564.0   Max.   :1.0000   Max.   :2.0000  
##  NA's   :10       NA's   :20                                       
##     thalach           exang           oldpeak         slope      
##  Min.   :  22.0   Min.   :0.0000   Min.   :0.00   Min.   :1.000  
##  1st Qu.: 133.0   1st Qu.:0.0000   1st Qu.:0.00   1st Qu.:1.000  
##  Median : 153.0   Median :0.0000   Median :0.80   Median :2.000  
##  Mean   : 161.9   Mean   :0.3267   Mean   :1.04   Mean   :1.601  
##  3rd Qu.: 166.0   3rd Qu.:1.0000   3rd Qu.:1.60   3rd Qu.:2.000  
##  Max.   :1710.0   Max.   :1.0000   Max.   :6.20   Max.   :3.000  
##  NA's   :10                                                      
##        ca                 thal                diag       
##  Min.   :-100000.0   Min.   :-100000.0   Min.   :0.0000  
##  1st Qu.:      0.0   1st Qu.:      3.0   1st Qu.:0.0000  
##  Median :      0.0   Median :      3.0   Median :0.0000  
##  Mean   :   -659.2   Mean   :   -655.4   Mean   :0.4587  
##  3rd Qu.:      1.0   3rd Qu.:      7.0   3rd Qu.:1.0000  
##  Max.   :      4.0   Max.   :      7.0   Max.   :1.0000  
## 
\end{verbatim}

Teniendo en cuenta el tipo de variables del dataset, se observa que
algunas varibles son tipo categoricas (\emph{sex},
\emph{ca},\emph{diag}) y otros son
numéricas(\emph{age},\emph{chol},\emph{\ldots{}}). Sin embargo, el
análisis previo muestra todas las variables como númericas.

\hypertarget{manipulaciuxf3n-de-las-variables}{%
\paragraph{Manipulación de las
variables}\label{manipulaciuxf3n-de-las-variables}}

\begin{Shaded}
\begin{Highlighting}[]
\CommentTok{\# Identificar datos faltantes}
\FunctionTok{sum}\NormalTok{(}\FunctionTok{is.na}\NormalTok{(heart))      }\CommentTok{\# Cuenta el total de NAs del dataset}
\end{Highlighting}
\end{Shaded}

\begin{verbatim}
## [1] 50
\end{verbatim}

\begin{Shaded}
\begin{Highlighting}[]
\FunctionTok{colSums}\NormalTok{(}\FunctionTok{is.na}\NormalTok{(heart))  }\CommentTok{\# Calcular el total de NA de cada variable }
\end{Highlighting}
\end{Shaded}

\begin{verbatim}
##       id      age      sex       cp trestbps     chol      fbs  restecg 
##        0       10        0        0       10       20        0        0 
##  thalach    exang  oldpeak    slope       ca     thal     diag 
##       10        0        0        0        0        0        0
\end{verbatim}

\hypertarget{identificaciuxf3n-de-outliers}{%
\subsubsection{\texorpdfstring{Identificación de
\emph{``outliers''}:}{Identificación de ``outliers'':}}\label{identificaciuxf3n-de-outliers}}

La variable trestbps(presión arterial) tiene un valor máx de 2000, lo
cual es atípico. Por esta razón debemos, reemplazar los outliers por la
media o por un valor más racional.

\begin{Shaded}
\begin{Highlighting}[]
\FunctionTok{library}\NormalTok{(outliers)        }\CommentTok{\# librería }
\FunctionTok{summary}\NormalTok{(heart)           }\CommentTok{\# Resumen de los datos}
\end{Highlighting}
\end{Shaded}

\begin{verbatim}
##        id             age             sex               cp       
##  Min.   :  1.0   Min.   : 2.00   Min.   :0.0000   Min.   :1.000  
##  1st Qu.: 76.5   1st Qu.:47.00   1st Qu.:0.0000   1st Qu.:3.000  
##  Median :152.0   Median :55.00   Median :1.0000   Median :3.000  
##  Mean   :152.0   Mean   :54.11   Mean   :0.6799   Mean   :3.158  
##  3rd Qu.:227.5   3rd Qu.:61.00   3rd Qu.:1.0000   3rd Qu.:4.000  
##  Max.   :303.0   Max.   :77.00   Max.   :1.0000   Max.   :4.000  
##                  NA's   :10                                      
##     trestbps           chol            fbs            restecg      
##  Min.   :  94.0   Min.   :126.0   Min.   :0.0000   Min.   :0.0000  
##  1st Qu.: 120.0   1st Qu.:212.0   1st Qu.:0.0000   1st Qu.:0.0000  
##  Median : 130.0   Median :243.0   Median :0.0000   Median :1.0000  
##  Mean   : 145.9   Mean   :247.8   Mean   :0.1485   Mean   :0.9901  
##  3rd Qu.: 140.0   3rd Qu.:277.0   3rd Qu.:0.0000   3rd Qu.:2.0000  
##  Max.   :2000.0   Max.   :564.0   Max.   :1.0000   Max.   :2.0000  
##  NA's   :10       NA's   :20                                       
##     thalach           exang           oldpeak         slope      
##  Min.   :  22.0   Min.   :0.0000   Min.   :0.00   Min.   :1.000  
##  1st Qu.: 133.0   1st Qu.:0.0000   1st Qu.:0.00   1st Qu.:1.000  
##  Median : 153.0   Median :0.0000   Median :0.80   Median :2.000  
##  Mean   : 161.9   Mean   :0.3267   Mean   :1.04   Mean   :1.601  
##  3rd Qu.: 166.0   3rd Qu.:1.0000   3rd Qu.:1.60   3rd Qu.:2.000  
##  Max.   :1710.0   Max.   :1.0000   Max.   :6.20   Max.   :3.000  
##  NA's   :10                                                      
##        ca                 thal                diag       
##  Min.   :-100000.0   Min.   :-100000.0   Min.   :0.0000  
##  1st Qu.:      0.0   1st Qu.:      3.0   1st Qu.:0.0000  
##  Median :      0.0   Median :      3.0   Median :0.0000  
##  Mean   :   -659.2   Mean   :   -655.4   Mean   :0.4587  
##  3rd Qu.:      1.0   3rd Qu.:      7.0   3rd Qu.:1.0000  
##  Max.   :      4.0   Max.   :      7.0   Max.   :1.0000  
## 
\end{verbatim}

\begin{Shaded}
\begin{Highlighting}[]
\FunctionTok{outlier}\NormalTok{(heart}\SpecialCharTok{$}\NormalTok{trestbps)  }\CommentTok{\# Identificar posibles valores atípicos de cada variable.}
\end{Highlighting}
\end{Shaded}

\begin{verbatim}
## [1] 2000
\end{verbatim}

\begin{Shaded}
\begin{Highlighting}[]
\FunctionTok{outlier}\NormalTok{(heart}\SpecialCharTok{$}\NormalTok{thalach)   }
\end{Highlighting}
\end{Shaded}

\begin{verbatim}
## [1] 1710
\end{verbatim}

\begin{Shaded}
\begin{Highlighting}[]
\FunctionTok{outlier}\NormalTok{(heart}\SpecialCharTok{$}\NormalTok{thal)}
\end{Highlighting}
\end{Shaded}

\begin{verbatim}
## [1] -1e+05
\end{verbatim}

\textbf{Preguntas:} * 1. ¿Cuáles variables del dataset presentan
outliers?\\
* 2. ¿Cuántos valores faltantes hay en el conjunto de datos
\emph{heart}? * 3. ¿En qué variables se concentran los valores
faltantes? * 4. ¿Cómo imputaría la media o la mediana de estos valores?

\hypertarget{identificar-outliers-usando-boxplot}{%
\subsubsection{Identificar Outliers usando
boxplot:}\label{identificar-outliers-usando-boxplot}}

\begin{Shaded}
\begin{Highlighting}[]
\CommentTok{\# variable trestbps}
\NormalTok{outlier\_v1 }\OtherTok{\textless{}{-}} \FunctionTok{boxplot}\NormalTok{(heart}\SpecialCharTok{$}\NormalTok{trestbps, }\AttributeTok{col=}\StringTok{"skyblue"}\NormalTok{, }\AttributeTok{frame.plot=} \ConstantTok{FALSE}\NormalTok{)}
\end{Highlighting}
\end{Shaded}

\includegraphics{11--EDA-Heart_files/figure-latex/unnamed-chunk-4-1.pdf}

\begin{Shaded}
\begin{Highlighting}[]
\NormalTok{outlier\_v1}\SpecialCharTok{$}\NormalTok{out}
\end{Highlighting}
\end{Shaded}

\begin{verbatim}
##  [1]  172 1500 1120  180 2000  174  178  192  180  178  180
\end{verbatim}

\begin{Shaded}
\begin{Highlighting}[]
\CommentTok{\# variable Thal}
\NormalTok{outlier\_v2 }\OtherTok{\textless{}{-}} \FunctionTok{boxplot}\NormalTok{(heart}\SpecialCharTok{$}\NormalTok{thal, }\AttributeTok{col=}\StringTok{"skyblue"}\NormalTok{, }\AttributeTok{frame.plot=} \ConstantTok{FALSE}\NormalTok{)}
\end{Highlighting}
\end{Shaded}

\includegraphics{11--EDA-Heart_files/figure-latex/unnamed-chunk-4-2.pdf}

\begin{Shaded}
\begin{Highlighting}[]
\NormalTok{outlier\_v2}\SpecialCharTok{$}\NormalTok{out   }\CommentTok{\# Cuantos}
\end{Highlighting}
\end{Shaded}

\begin{verbatim}
## [1] -1e+05 -1e+05
\end{verbatim}

\begin{Shaded}
\begin{Highlighting}[]
\CommentTok{\# Entonces para eliminar los outliers usamos el operador pertenece \%in\% que funciona igual que el símbolo matemático ∈ que se usa en la teoría de conjuntos.}
\CommentTok{\# heart \textless{}{-}heart[!(heart$trestbps \%in\% outlier\_var$out),]  \# Eliminar los outliers}
\end{Highlighting}
\end{Shaded}

\hypertarget{manejo-de-outliers}{%
\subsubsection{Manejo de Outliers:}\label{manejo-de-outliers}}

\begin{Shaded}
\begin{Highlighting}[]
\CommentTok{\# variable thalach  }
\CommentTok{\# La variable *thalach* tiene un valor atípico de 1710 y 22:}
\NormalTok{heart}\SpecialCharTok{$}\NormalTok{thalach }\OtherTok{\textless{}{-}} \FunctionTok{ifelse}\NormalTok{(heart}\SpecialCharTok{$}\NormalTok{thalach }\SpecialCharTok{\textgreater{}=} \DecValTok{250} \SpecialCharTok{|}\NormalTok{ heart}\SpecialCharTok{$}\NormalTok{thalach }\SpecialCharTok{\textless{}} \DecValTok{30}\NormalTok{,}
                        \FunctionTok{mean}\NormalTok{(heart}\SpecialCharTok{$}\NormalTok{thalach),heart}\SpecialCharTok{$}\NormalTok{thalach)}

\CommentTok{\# variable thal: 3: normal, 6 fixed detect, 7:reversable defect  }

\NormalTok{heart}\SpecialCharTok{$}\NormalTok{thal }\OtherTok{\textless{}{-}} \FunctionTok{factor}\NormalTok{(heart}\SpecialCharTok{$}\NormalTok{thal)}
\FunctionTok{str}\NormalTok{(heart}\SpecialCharTok{$}\NormalTok{thal)}
\end{Highlighting}
\end{Shaded}

\begin{verbatim}
##  Factor w/ 4 levels "-1e+05","3","6",..: 3 2 4 2 2 2 2 2 4 4 ...
\end{verbatim}

\begin{Shaded}
\begin{Highlighting}[]
\NormalTok{heart}\SpecialCharTok{$}\NormalTok{thal }\OtherTok{\textless{}{-}} \FunctionTok{ifelse}\NormalTok{(heart}\SpecialCharTok{$}\NormalTok{thal }\SpecialCharTok{\textless{}=} \SpecialCharTok{{-}}\DecValTok{700}\NormalTok{,}
                        \FunctionTok{mode}\NormalTok{(heart}\SpecialCharTok{$}\NormalTok{thal),heart}\SpecialCharTok{$}\NormalTok{thal)}
\end{Highlighting}
\end{Shaded}

\begin{verbatim}
## Warning in Ops.factor(heart$thal, -700): '<=' not meaningful for factors
\end{verbatim}

\begin{Shaded}
\begin{Highlighting}[]
\CommentTok{\# Imputación de mean en NAs }
\CommentTok{\#heart$age[is.na(heart$age)]\textless{}{-} round(mean(heart$age, na.rm = TRUE))}
\CommentTok{\#heart$chol[is.na(heart$chol)]\textless{}{-} round (mean(heart$chol, na.rm = TRUE))}
\CommentTok{\#heart$trestbps[is.na(heart$chol)]\textless{}{-} round (mean(heart$chol, na.rm = TRUE))}
\CommentTok{\#heart$chol[is.na(heart$chol)]\textless{}{-} round (mean(heart$chol, na.rm = TRUE))}


\CommentTok{\#heart\_mod \textless{}{-} heart \%\textgreater{}\%}
\CommentTok{\#  dplyr::select({-}id)\%\textgreater{}\%     \# Eliminar Variable id }
\end{Highlighting}
\end{Shaded}

\textbf{Avanzado:} Creando un ciclo for

\begin{Shaded}
\begin{Highlighting}[]
\ControlFlowTok{for}\NormalTok{(i }\ControlFlowTok{in} \DecValTok{1}\SpecialCharTok{:}\FunctionTok{ncol}\NormalTok{(heart))\{}
\NormalTok{  heart[}\FunctionTok{is.na}\NormalTok{(heart[,i]), i] }\OtherTok{\textless{}{-}} \FunctionTok{mean}\NormalTok{(heart[,i], }\AttributeTok{na.rm =} \ConstantTok{TRUE}\NormalTok{)}
\NormalTok{    \}}
\FunctionTok{summary}\NormalTok{(heart)}
\end{Highlighting}
\end{Shaded}

\begin{verbatim}
##        id             age             sex               cp       
##  Min.   :  1.0   Min.   : 2.00   Min.   :0.0000   Min.   :1.000  
##  1st Qu.: 76.5   1st Qu.:47.50   1st Qu.:0.0000   1st Qu.:3.000  
##  Median :152.0   Median :55.00   Median :1.0000   Median :3.000  
##  Mean   :152.0   Mean   :54.11   Mean   :0.6799   Mean   :3.158  
##  3rd Qu.:227.5   3rd Qu.:61.00   3rd Qu.:1.0000   3rd Qu.:4.000  
##  Max.   :303.0   Max.   :77.00   Max.   :1.0000   Max.   :4.000  
##                                                                  
##     trestbps           chol            fbs            restecg      
##  Min.   :  94.0   Min.   :126.0   Min.   :0.0000   Min.   :0.0000  
##  1st Qu.: 120.0   1st Qu.:214.0   1st Qu.:0.0000   1st Qu.:0.0000  
##  Median : 130.0   Median :246.0   Median :0.0000   Median :1.0000  
##  Mean   : 145.9   Mean   :247.8   Mean   :0.1485   Mean   :0.9901  
##  3rd Qu.: 140.0   3rd Qu.:274.0   3rd Qu.:0.0000   3rd Qu.:2.0000  
##  Max.   :2000.0   Max.   :564.0   Max.   :1.0000   Max.   :2.0000  
##                                                                    
##     thalach          exang           oldpeak         slope      
##  Min.   : 71.0   Min.   :0.0000   Min.   :0.00   Min.   :1.000  
##  1st Qu.:136.0   1st Qu.:0.0000   1st Qu.:0.00   1st Qu.:1.000  
##  Median :152.0   Median :0.0000   Median :0.80   Median :2.000  
##  Mean   :149.4   Mean   :0.3267   Mean   :1.04   Mean   :1.601  
##  3rd Qu.:165.0   3rd Qu.:1.0000   3rd Qu.:1.60   3rd Qu.:2.000  
##  Max.   :202.0   Max.   :1.0000   Max.   :6.20   Max.   :3.000  
##                                                                 
##        ca                 thal          diag       
##  Min.   :-100000.0   Min.   : NA   Min.   :0.0000  
##  1st Qu.:      0.0   1st Qu.: NA   1st Qu.:0.0000  
##  Median :      0.0   Median : NA   Median :0.0000  
##  Mean   :   -659.2   Mean   :NaN   Mean   :0.4587  
##  3rd Qu.:      1.0   3rd Qu.: NA   3rd Qu.:1.0000  
##  Max.   :      4.0   Max.   : NA   Max.   :1.0000  
##                      NA's   :303
\end{verbatim}

Note that the \texttt{echo\ =\ FALSE} parameter was added to the code
chunk to prevent printing of the R code that generated the plot.

\end{document}
